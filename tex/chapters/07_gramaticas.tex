\section{Gramáticas}\label{c:gramaticas}
Um das funcionalidades interessantes do sistema é a possibilidade de adicionar eventos e notícias de forma textual, isto é, introduzindo um texto estruturado com as informações necessárias à inserção da informação na base de dados. Esta opção está disponível na secção \textit{News management} com a opção \textit{Insert Articles by Grammar} e na secção \textit{Agenda management} com a opção \textit{Insert Events By Grammar}.
Para esta poder ser completada com sucesso, isto é, a informação ser extraída corretamente e serem criados eventos/notícias, o texto introduzido é verificado quanto à sua correção sintática de acordo com as gramáticas elaboradas. Também é importante referir que estas funcionalidades apenas estão disponíveis para o administrador.


\subsection{Agenda}
Para a criação da funcionalidade de adicionar eventos \textit{by grammar}, começamos por criar o ficheiro com a gramática de atributos utilizada na verificação. Nesse sentido, analisamos os \textit{Schemas} disponíveis na pasta \texttt{Models}, nomeadamente os relativos ao artigos de notícias e aos eventos da agenda. Isto permitiu-nos saber que atributos incluir na gramática e também quais são ou não \textit{required} para definir corretamente as produções.

A gramática de atributos criada tem o seu código escrito em \textit{JavaScript} e possui 14 produções (\texttt{agenda,} \texttt{event,} \texttt{title,} \texttt{desc,} \texttt{local,} \texttt{time,} \texttt{start,} \texttt{end,} \texttt{start,} \texttt{end,} \texttt{startDate,} \texttt{startHour,} \texttt{endDate} e \texttt{endHour}).
Fazemos algumas verificações, nomeadamente a correção temporal, isto é, se a a data marcada para o fim do evento é efetivamente depois da data inicial (caso o evento seja num só dia, verifica se a hora de fim é posterior à hora de início).

Segue-se um excerto da gramática em questão.
\begin{framed}
\begin{lstlisting}[language=ANTLR]
grammar agenda;

agenda returns [var val, var errors]
@init {
     $val = []
     $errors = []
}   
     : 'AGENDA:' (
		event {
 				if($event.error!="") $errors.push($event.error);
				else $val.push($event.val);
               } ';'
	)+;

event returns[var val, var error]
@init {
     var existsDesc = false
     var existsLocal = false
}
	: title (desc {existsDesc=true})? (local {existsLocal=true})? time 
     {    
          if($time.ts.sdate>$time.te.edate){
               $val = ""
               $error = "Start date after end date on "+$title.val+" event!"
          }else{
               if ($time.ts.sdate==$time.te.edate && $time.ts.shour>$time.te.ehour ){
                    $val = ""
                    $error = "Start hour after end hour on "+$title.val+" event!"     
               }else {
                    $val = {
                         title: $title.val,
                         startDate: $time.ts.sdate,
                         startHour: $time.ts.shour,
                         endDate: $time.te.edate,
                         endHour: $time.te.ehour
                    }
                    if (existsDesc) $val.desc = $desc.val
                    else $val.desc = ""
                    if (existsLocal) $val.local = $local.val
                    else $val.local = ""
                    $error = ""
               }
          }
     }
     ;
\end{lstlisting}
\end{framed}
\begin{center}
\textit{Excerto do ficheiro} \texttt{agenda.g4} \textit{da pasta} \texttt{grammars}.
\end{center}

No excerto acima é possível ver a verificação da data e a criação da mensagem de erro, quando esta se aplica. É ainda possível ver que a gramática permite a adição de vários eventos de uma só vez. Assim, o resultado de tentar adicionar os eventos do ficheiro abaixo seria  inserção correta destes, visto que respeitam a estrutura definida na gramática sendo portanto um exemplo sintaticamente correto. Por outro lado, a nível semântico, também não acarreta problemas já que as datas/horas seguem as regras definidas.

\begin{framed}
\begin{lstlisting}[language=ANTLR]
    AGENDA:

    TITLE: 'Festa das Colheitas'
    DESC: 'Celebração do mundo rural: agricultura, gastronomia, tradições.'
    LOCAL: 'Vila Verde'
    BEGINS: 2019-10-03 09:00
    ENDS: 2019-10-07 23:00
    ;
    
    TITLE: 'Santo Isidro'
    BEGINS: 2019-09-30 10:00
    ENDS: 2019-09-30 11:00
    ;
\end{lstlisting}
\end{framed}
\begin{center}
\textit{Ficheiro} \texttt{agenda\_right.g4} \textit{da pasta} \texttt{grammars/test\_files}.
\end{center}

No entanto, o ficheiro abaixo não irá resultar na inserção do evento nele definido. Apesar de os campos obrigatórios se encontrarem todos definidos, a nível semântico é possível verificar que o \textit{time} definido não faz sentido uma vez que a hora de fim do evento é posterior à hora de início (sendo que o evento é de um só dia). Assim, o evento não pode ser inserido e resulta na mensagem de erro \textit{"Start hour after end hour on ``Santo Isidro'' event!"}

\begin{framed}
\begin{lstlisting}[language=ANTLR]
    AGENDA:

    TITLE: 'Santo Isidro'
    BEGINS: 2019-09-30 10:00
    ENDS: 2019-09-30 08:00
    ;
\end{lstlisting}
\end{framed}
\begin{center}
\textit{Excerto do ficheiro} \texttt{agenda\_wrong.g4} \textit{da pasta} \texttt{grammars/test\_files}.
\end{center}

O exemplo abaixo também não pode ser inserido já que não contém dois dos campos obrigatórios (\texttt{BEGINS:} e \texttt{ENDS:}), nomeadamente data e hora de início e fim. Assim, falha devido a erros sintáticos, apesar do título, descrição e local estarem bem definidos.

\begin{framed}
\begin{lstlisting}[language=ANTLR]
    AGENDA:

    TITLE: 'Festa São Miguel'
    DESC: 'Banda toca na procissão.'
    LOCAL: 'Prado São Miguel'
    ;
\end{lstlisting}
\end{framed}
\begin{center}
\textit{Excerto do ficheiro} \texttt{agenda\_wrong.g4} \textit{da pasta} \texttt{grammars/test\_files}.
\end{center}

Por fim, a tentativa de inserção do evento '\textit{Festas das Colheitas}' irá falhar também devido erros sintáticos como o anterior.
Efetivamente, todos os campos (\texttt{TITLE}, \texttt{DESC}, \texttt{BEGINS}, \texttt{ENDS} e \texttt{LOCAL}) estão corretamente definidos, e a datas/horas seguem as regras estipuladas. No entanto, não estão pela ordem esperada (definida na gramática), logo resultam em erros sintáticos.

\begin{framed}
\begin{lstlisting}[language=ANTLR]
    AGENDA:

    TITLE: 'Festa das Colheitas'
    BEGINS: 2019-10-03 09:00
    ENDS: 2019-10-07 23:00
    DESC: 'Celebração do mundo rural: agricultura, gastronomia, tradições.'
    LOCAL: 'Vila Verde'
    ;
\end{lstlisting}
\end{framed}
\begin{center}
\textit{Excerto do ficheiro} \texttt{agenda\_wrong.g4} \textit{da pasta} \texttt{grammars/test\_files}.
\end{center}

\subsection{News}
A opção \textit{Add Article By Grammar} foi concebida, tal como a sua semelhante dos eventos, tendo como base o seu ficheiro correspondente nos \texttt{Models}, concretamente, o \textit{schema} definido no \texttt{article.js}. A gramática em questão possui 10 produções: \texttt{newspaper}, \texttt{news}, \texttt{titles}, \texttt{title}, \texttt{topics}, \texttt{topic}, \texttt{body}, \texttt{date}, \texttt{authors} e \texttt{author}.

A nível de verificações efetuadas, é conferida a data da notícia e, caso essa seja posterior ao dia atual, a inserção falha com o aviso \textit{Date is in the future on X article}. 

Segue-se um excerto da gramática.

\begin{framed}
\begin{lstlisting}[language=ANTLR]
grammar news;

newspaper returns [var val, var errors]
@init{
	$val = []
	$errors = []
}
		 :'NEWS:' (news {
			 if($news.error!="") $errors.push($news.error);
			 else $val.push($news.val);
		 } ';')+
		 ;

news returns [var val, var error]
@init{
	function today(){
		var today = new Date();
		var dd = today.getDate();
		var mm = today.getMonth() + 1; //January is 0!
		var yyyy = today.getFullYear();

		if (dd < 10) {
			dd = '0' + dd;
		}

		if (mm < 10) {
			mm = '0' + mm;
		}

		today = yyyy + '-' + mm + '-' + dd;
		return today;
	}
	var existTopics = false
	var existAuthors = false
}
	: titles (topics {existTopics = true})? body date (authors {existAuthors = true})?
	{
		if(today()>=$date.val){
			$val = {
				title: $titles.titleOut,
				subtitle: $titles.subtitle,
				body: $body.val,
				date: $date.val
			}
			if(existTopics) $val.topics = $topics.val
			else $val.topics = []
			if(existAuthors) $val.authors = $authors.val
			else $val.authors = []
			$error = ""
		}else{
			$val = ""
			$error = "Date is in the future on " + $titles.titleOut + " article!"
		}
	}
    ;
\end{lstlisting}
\end{framed}
\begin{center}
\textit{Excerto do ficheiro} \texttt{news.g4} \textit{da pasta} \texttt{grammars}.
\end{center}


Segue-se um exemplo que resulta na adição à coleção dos artigos de notícias, visto que ambas as notícias definidas estão corretas a nível sintático e semântico, contendo a primeira todos os campos possíveis preenchidos e a segunda apenas os de preenchimento obrigatório (\texttt{TITLE}, \texttt{BODY} e \texttt{DATE}).
\begin{framed}
\begin{lstlisting}[language=ANTLR]
    NEWS:
    
    TITLE: 'Banda Filarmónica da UM'
    SUBTITLE: 'Recorde de Concertos Ultrassado'
    TOPICS: 'Universidade do Minho' , 'concertos'
    BODY: 'A banda filarmónica da Universidade do Minho bateu este mês o recorde de concertos por ano.'
    DATE: 2018-12-27
    AUTHORS: 'D. Barbosa' , 'JC Martins'
    ;
    
    TITLE: 'A melhor Banda Filarmónica'
    BODY: 'Campeã do Mundial de Bandas Filarmónicas.'
    DATE: 2018-01-12
    ;
\end{lstlisting}
\end{framed}
\begin{center}
\textit{Excerto do ficheiro} \texttt{news\_right.txt} \textit{da pasta} \texttt{grammars/test\_files}.
\end{center}

No entanto, em alguns casos não é possível adicionar os artigos tal como pretendido. Um exemplo no qual isso acontece é quando ocorrem erros semânticos relativos à data. Exemplificando, não tem muito nexo escrever uma notícia com data posterior ao dia atual. Tal como mencionado previamente, este é um caso cuja verificação está definida na gramática. Seguindo esta lógica, a notícia abaixo (de título \textit{"Banda Filarmónica da UM"}) não poderá ser adicionada antes do dia 27 de março de 2019.
\begin{framed}
\begin{lstlisting}[language=ANTLR]
    NEWS:
    
    TITLE: 'Banda Filarmónica da UM'
    BODY: 'A banda filarmónica da Universidade do Minho bateu este mês o recorde de concertos por ano.'
    DATE: 2019-03-27
    ;
\end{lstlisting}
\end{framed}
\begin{center}
\textit{Excerto do ficheiro} \texttt{news\_wrong.txt} \textit{da pasta} \texttt{grammars/test\_files}.
\end{center}

Outro caso que não resulta numa correta inserção do artigo, está relacionado com erros sintáticos. No caso, a ausência do preenchimento de campos de natureza \textit{required}, nomeadamente \textit{DATE}.
\begin{framed}
\begin{lstlisting}[language=ANTLR]
    NEWS:
    
    TITLE: 'A melhor Banda Filarmónica'
    BODY: 'Campeã do Mundial de Bandas Filarmónicas.'
    AUTHORS: 'D. Barbosa' , 'JC Martins'
    ;
\end{lstlisting}
\end{framed}
\begin{center}
\textit{Excerto do ficheiro} \texttt{news\_wrong.txt} \textit{da pasta} \texttt{grammars/test\_files}.
\end{center}

Por fim, outro exemplo de erros sintáticos que impedem o funcionamento correto do processo \textit{Add News By Grammar"} é a troca das posições dos campos na notícia o que impede um reconhecimento adequado da produção. No caso abaixo ilustrado, o campo \texttt{SUBTITLE} que deveria estar imediatamente após o \texttt{TITLE} encontra-se a seguir ao campo \texttt{AUTHORS}, entre outras incoerências.
\begin{framed}
\begin{lstlisting}[language=ANTLR]
    NEWS:
    
    TITLE: 'Banda Filarmónica da UM'
    DATE: 2018-12-27
    AUTHORS: 'D. Barbosa' , 'JC Martins'
    SUBTITLE: 'Recorde de Concertos Ultrassado'
    TOPICS: 'Universidade do Minho' , 'concertos'
    BODY: 'A banda filarmónica da Universidade do Minho bateu este mês o recorde de concertos por ano.'
    ;
\end{lstlisting}
\end{framed}
\begin{center}
\textit{Excerto do ficheiro} \texttt{news\_wrong.txt} \textit{da pasta} \texttt{grammars/test\_files}.
\end{center}


\subsection{Interligação entre Gramáticas e Web site}

Tudo isto é possível devido ao \textit{Gradle Build Tool}, um sistema de automação de compilação \textit{open-source} com vários plugins, sendo que o plugin utilizado é o \texttt{java} de forma a gerar código javascript, em vez de código java, a ser utilizado pelo sistema (iBanda) desenvolvido.

\begin{framed}
\begin{lstlisting}[language=gradle]
apply plugin: 'java'
 
repositories {
    jcenter()
}
 
dependencies {
    runtime 'org.antlr:antlr4:4.5.2'
}
 
task generateAgendaParser(type:JavaExec) {
   main = 'org.antlr.v4.Tool'
   classpath = sourceSets.main.runtimeClasspath
   args = ['-Dlanguage=JavaScript', 'agenda.g4', '-o', 'agenda']
}

task generateNewsParser(type:JavaExec) {
   main = 'org.antlr.v4.Tool'
   classpath = sourceSets.main.runtimeClasspath
   args = ['-Dlanguage=JavaScript', 'news.g4', '-o', 'news']
}
\end{lstlisting}
\end{framed}
\begin{center}
\textit{Ficheiro} \texttt{build.gradle} \textit{da pasta} \texttt{Grammars}~\cite{AW}.
\end{center}

Para além disso, já no sistema desenvolvido é usado a package \texttt{antlr4/index} disponível em \textit{JavaScript}, por forma a utilizar o código gerado pelo Gradle. 

O excerto seguinte de código realiza a verificação léxica e sintática do input para o caso da agenda:

\begin{framed}
\begin{lstlisting}[language=JavaScript]
var NewsLexer = require('../grammars/news/newsLexer').newsLexer
var NewsParser = require('../grammars/news/newsParser').newsParser
var NewsListener = require('../grammars/news/newsListener').newsListener

.
.
.

//recebe o input e transforma-o num tipo reconhecido pelo antlr4
var chars = new antlr4.InputStream(req.body.grammar);
//cria o lexer a usar, a partir do código gerado pelo gradle
var lexer = new AgendaLexer(chars);
var tokens  = new antlr4.CommonTokenStream(lexer);
//cria o parser a usar, a partir do código gerado pelo gradle
var parser = new AgendaParser(tokens);

//variável que irá guardar os erros léxicos gerados pela gramática
var log = []

//trocado o console.error por forma a capturar os erros gerados pelo analisador léxico
var exLogError = console.error
console.error = function(msg) {
    log.push(msg)
}

parser.buildParseTrees = true;   
//cria-se a àrvore ao chamar a regra onde se começa (neste caso começa-se pela regra agenda)
var tree = parser.agenda();
//criação do listener
var agendaListener = new AgendaListener();
//é percorrida a árvore criada com o listener associado (no listener podia ser associado código à entrada e à saída das regras, o listener usado é o default que não possui nenhum código nessas fases)
antlr4.tree.ParseTreeWalker.DEFAULT.walk(agendaListener, tree);

//reset do console.error
console.error = exLogError
\end{lstlisting}
\end{framed}
\begin{center}
\textit{Exemplo de Interligação para a Agenda entre o código gerado pelo ANTLR a partir da gramática produzida e o sistema desenvolvido.}~\cite{AMT,MA,DOCA}
\end{center}

De forma a obter o que foi construído pelo parser, existem dois atributos sintetizados na regra inicial a partir dos quais é possível obter os artigos (\verb|var val|) e os erros (\verb|var errors|). Estes atributos são acessíveis após percorrer a árvore fazendo \verb|tree.val| para obter \verb|val| e \verb|tree.errors| para obter o \verb|errors|.

Tendo isto em conta, caso hajam erros quer léxicos (\verb|var log|) quer sintáticos (\verb|var errors|), serão apresentados ao user. Caso não hajam erros, são inseridos na base de dados os eventos.

Para as \textit{news} é feito de forma semelhante, mudando apenas o facto de que em vez de se usarem os lexer's, parser's e listener's do \texttt{agenda} usa-se do news, bem como a regra inicial que é \verb|newspaper| em vez de \verb|agenda|.