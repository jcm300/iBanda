\begin{abstract}
O projeto prático sobre o qual este relatório incide tem como nome \textit{iBanda - Arquivo Digital Musical} e consiste numa aplicação \textit{web} que implementa um repositório digital de obras musicais e respetivas partituras, respeitando a estrutura do modelo de referência internacional OAIS ("Open Archive Information System"). O projeto destina-se a ser usado por uma banda filarmónica para aceder às partituras, bastanto apenas terem acesso à \textit{internet} e uma conta válida.

O sistema permite ainda a gestão de uma agenda de eventos, gestão de notícias, gestão dos utilizadores/músicos da
banda, gestão de repertórios e gestão de uma pequena biblioteca de suporte.

No primeiro capítulo, fazemos uma breve introdução ao tema. No capítulo \ref{c:models} detalhamos os \textit{schemas} desenvolvidos e decisões tomadas relativamente à base de dados, no capítulo \ref{c:controllers} explicamos algumas das funções criadas na componente dos \textit{controllers}, no capítulo \ref{c:routes} incidimos sobre os vários \textit{routes}, no capítulo \ref{c:views} falamos sobre a nossa abordagem tomada relativamente às \textit{views}, no capítulo \ref{c:localFichs} identificamos as várias localizações dos ficheiros que são armazenados, no capítulo \ref{c:gramaticas} detalhamos o processo de criação das gramáticas criadas, no capítulos \ref{c:instalacao} explicamos como fazer a instalação do sistema e, por fim, no capítulo \ref{c:conclusoes} mencionamos algumas conclusões a que chegamos após a realização deste trabalho.
\end{abstract}

\section{IMPORTANTE PARA PRC}
\begin{itemize}
{\color{red}
    \item secção do relatório de funcionalidades a destacar
    \item imprimir coisas mesmo importantes a mostrar na apresentação
    \item imprimir 2 relatórios (if needed)}
\end{itemize}