\section{Conclusões}\label{c:conclusoes}
\qquad As aplicações \textit{web} que podem ser criadas hoje em dia têm um enorme potencial para ajudar negócios, pessoas e organizações. Neste caso, o público alvo são bandas filarmónicas, no entanto, o sistema é facilmente adaptável a uma grande variedade de contextos. A nível de tecnologias, ferramentas e conceitos de que tiramos partido, destacam-se os modelos \textit{MVC} da aplicação, \textit{OAIS} do repositório, e o \textit{npm}.

Com este trabalho, o grupo conseguiu aprimorar as suas competências no uso de \textit{npm}, ferramentas novas como \textit{Gradle Build Tool}, mas também a nível de programação em \textit{JavaScript}. Conseguiu ainda desenvolver pensamento crítico devido aos desafios que foram surgindo (como por exemplo, a ligação das gramáticas ao sistema) e, em suma, pôr em prática os conhecimentos adquiridos ao longo do semestre.

No entanto, o trabalho apresenta alguns pontos a melhorar, entre os quais, avisar por email quando a conta é aprovada, quando forem alteradas as suas permissões ou quando a conta é apagada. Por outro lado, permitir a inserção de texto na parte gramatical a partir de um ficheiro seria também uma funcionalidade extra interessante.
Por fim, o outro ponto a melhorar seria o acréscimo da funcionalidade da enciclopédia do material armazenado que consiste na busca de informação à \textit{internet} sobre os autores e as obras armazenadas.

Assim, em conclusão, fazemos uma avaliação positiva a nível das funcionalidades implementadas e objetivos cumpridos.
